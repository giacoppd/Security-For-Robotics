\documentclass[IEEEtran,letterpaper,10pt,titlepage,draftclsnofoot,onecolumn]{article}

\usepackage{nopageno}
\usepackage{alltt}
\usepackage{float}
\usepackage{color}
\usepackage{url}
\usepackage{balance}
\usepackage{enumitem}
\usepackage{pstricks, pst-node}
\usepackage{geometry}
\geometry{textheight=9.5in, textwidth=7in}
\newcommand{\cred}[1]{{\color{red}#1}}
\newcommand{\cblue}[1]{{\color{blue}#1}}
\usepackage{hyperref}
\usepackage{textcomp}
\usepackage{listings}

\definecolor{dkgreen}{rgb}{0,0.6,0}
\definecolor{gray}{rgb}{0.5,0.5,0.5}
\definecolor{mauve}{rgb}{0.58,0,0.82}
\lstset{frame=tb,
  language=c,
  aboveskip=3mm,
  belowskip=3mm,
  showstringspaces=false,
  columns=flexible,
  basicstyle={\small\ttfamily},
  numbers=none,
  numberstyle=\tiny\color{gray},
  keywordstyle=\color{blue},
  commentstyle=\color{dkgreen},
  stringstyle=\color{mauve},
  breaklines=true,
  breakatwhitespace=true,
  tabsize=3
}

\def\name{Zach Rogers}

\begin{document}
\begin{titlepage}
  \begin{center}
    \vspace*{1cm}

    \huge
    \textbf{Technology Review - Security for Robotics}
  \vspace{0.5cm}

    \textit{Zach Rogers}\\
  \vspace{0.5cm}
    \vfill
    \large
    \textbf{CS461 Capstone}\\
  \vspace{5mm}

    \textbf{16 Nov 2016}\\

    \vfill
    \end{center}
\end{titlepage}

\begin{abstract}
Our goal as a group is to identify vulnerabilities, both hardware and software related, within our drone system.
A big part of that will have to do with the drone's communication channel, which describes how a user controls a
drone during flight and general operation. In order to attack the communication channel, we must first understand
how the drones communicate with the user, and how the user sends commands to the drone. This will involve lots of
data capturing. So my focus right now is to determine how we will be capturing that data, and how we will use that
data to reverse-engineer the drone's methods of communication for the purpose of developing attack methods.
\end{abstract}

\hrule\vspace{5mm}
\subsection*{Drone Communication Channel}
The two drones that we have use a 2.4Ghz data-link between the drone and the receiver ground-station unit.
That receiver unit then uses a Bluetooth connection to connect to the user's controller, which is a physical controller
or device such as a laptop or tablet.\cite{NazaM2} With this in mind, there are two communication channels that can be
targeted. The 2.4Ghz frequency is commonly used by most Wireless Access Points, following the 802.11a/b/g
standard.Bluetooth is another method of communication via RF (radio frequencies). This means that communication packets
for the drone is being sent and received out in the open, which can be can be intercepted and analyzed with the right
tools.

\subsection*{Methods of Data Capture}


\subsection*{Leveraging Captured Data to Develop Attack Methods}


\newpage
\bibliographystyle{IEEEtran}
\bibliography{tech_review}

\end{document}
