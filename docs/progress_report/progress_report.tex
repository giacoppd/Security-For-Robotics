\documentclass[IEEEtran,letterpaper,10pt,notitlepage,draftclsnofoot,onecolumn]{article}

\usepackage{nopageno}
\usepackage{alltt}
\usepackage{float}
\usepackage{color}
\usepackage{url}
\usepackage{balance}
\usepackage{enumitem}
\usepackage{pstricks, pst-node}
\usepackage{geometry}
\geometry{textheight=9.5in, textwidth=7in}
\newcommand{\cred}[1]{{\color{red}#1}}
\newcommand{\cblue}[1]{{\color{blue}#1}}
\usepackage{hyperref}
\usepackage{textcomp}
\usepackage{listings}
\usepackage{titling}
\usepackage{graphicx}
\usepackage{url}
\usepackage{setspace}

\definecolor{dkgreen}{rgb}{0,0.6,0}
\definecolor{gray}{rgb}{0.5,0.5,0.5}
\definecolor{mauve}{rgb}{0.58,0,0.82}
\lstset{frame=tb,
  language=c,
  aboveskip=3mm,
  belowskip=3mm,
  showstringspaces=false,
  columns=flexible,
  basicstyle={\small\ttfamily},
  numbers=none,
  numberstyle=\tiny\color{gray},
  keywordstyle=\color{blue},
  commentstyle=\color{dkgreen},
  stringstyle=\color{mauve},
  breaklines=true,
  breakatwhitespace=true,
  tabsize=3
}

% 1. Fill in these details
\def \CapstoneTeamName{   RoboSec}
\def \CapstoneTeamNumber{   50}
\def \GroupMemberOne{     Emily Longman}
\def \GroupMemberTwo{     Zach Rogers}
\def \GroupMemberThree{     Dominic Giacoppe}
\def \CapstoneProjectName{    Security for Robotics}
\def \CapstoneSponsorCompany{ Oregon State University}
\def \CapstoneSponsorPerson{    Vedanth Narayanan}

% 2. Uncomment the appropriate line below so that the document type works
\def \DocType{    %Problem Statement
        %Requirements Document
        %Technology Review
        %Design Document
        Progress Report
        }
      
\newcommand{\NameSigPair}[1]{\par
\makebox[2.75in][r]{#1} \hfil   \makebox[3.25in]{\makebox[2.25in]{\hrulefill} \hfill    \makebox[.75in]{\hrulefill}}
\par\vspace{-12pt} \textit{\tiny\noindent
\makebox[2.75in]{} \hfil    \makebox[3.25in]{\makebox[2.25in][r]{Signature} \hfill  \makebox[.75in][r]{Date}}}}
% 3. If the document is not to be signed, uncomment the RENEWcommand below
%\renewcommand{\NameSigPair}[1]{#1}

%%%%%%%%%%%%%%%%%%%%%%%%%%%%%%%%%%%%%%%
\begin{document}
\begin{titlepage}
    \pagenumbering{gobble}
    \begin{singlespace}
      \includegraphics[height=4cm]{coe_v_spot1}
        \hfill 
        % 4. If you have a logo, use this includegraphics command to put it on the coversheet.
        %\includegraphics[height=4cm]{CompanyLogo}   
        \par\vspace{.2in}
        \centering
        \scshape{
            \huge CS Capstone \DocType \par
            {\large\today}\par
            \vspace{.5in}
            \textbf{\Huge\CapstoneProjectName}\par
            \vfill
            {\large Prepared for}\par
            \Huge \CapstoneSponsorCompany\par
            \vspace{10pt}
            {\Large\NameSigPair{\CapstoneSponsorPerson}\par}
            {\large Prepared by }\par
            Group\CapstoneTeamNumber\par
            % 5. comment out the line below this one if you do not wish to name your team
            \CapstoneTeamName\par 
            \vspace{10pt}
            {\Large
                \NameSigPair{\GroupMemberOne}\par
                \NameSigPair{\GroupMemberTwo}\par
                \NameSigPair{\GroupMemberThree}\par
            }
            \vspace{20pt}
        }
        \begin{abstract}
          In drones and other networked robotics there is a broad array of security vulnerabilities that can be leveraged in an attack.
          We will evaluate the ROS to find as many of these security holes as we can and document them.
          The different vulnerabilities found will be categorized into malware, sensor hacks, network and control channel attacks, and physical breaches.
          For some of these exploits we may be able to implement solutions, which will also be documented.
          These findings and any solutions will be added to an ongoing academic effort to make robotics more secure.
        \end{abstract}     
    \end{singlespace}
\end{titlepage}
\newpage
\pagenumbering{arabic}
\tableofcontents
% 7. uncomment this (if applicable). Consider adding a page break.
%\listoffigures
%\listoftables
\clearpage

\section{Introduction}
Progress Report here...

\bibliographystyle{IEEEtran}
\bibliography{progress_report.bib}

\end{document}