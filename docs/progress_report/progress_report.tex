\documentclass[IEEEtran,letterpaper,10pt,notitlepage,draftclsnofoot,onecolumn]{article}

\usepackage{nopageno}
\usepackage{alltt}
\usepackage{float}
\usepackage{color}
\usepackage{url}
\usepackage{balance}
\usepackage{enumitem}
\usepackage{pstricks, pst-node}
\usepackage{geometry}
\geometry{textheight=9.5in, textwidth=7in}
\newcommand{\cred}[1]{{\color{red}#1}}
\newcommand{\cblue}[1]{{\color{blue}#1}}
\usepackage{hyperref}
\usepackage{textcomp}
\usepackage{listings}
\usepackage{titling}
\usepackage{graphicx}
\usepackage{url}
\usepackage{setspace}

\definecolor{dkgreen}{rgb}{0,0.6,0}
\definecolor{gray}{rgb}{0.5,0.5,0.5}
\definecolor{mauve}{rgb}{0.58,0,0.82}
\lstset{frame=tb,
  language=c,
  aboveskip=3mm,
  belowskip=3mm,
  showstringspaces=false,
  columns=flexible,
  basicstyle={\small\ttfamily},
  numbers=none,
  numberstyle=\tiny\color{gray},
  keywordstyle=\color{blue},
  commentstyle=\color{dkgreen},
  stringstyle=\color{mauve},
  breaklines=true,
  breakatwhitespace=true,
  tabsize=3
}

% 1. Fill in these details
\def \CapstoneTeamName{   RoboSec}
\def \CapstoneTeamNumber{   50}
\def \GroupMemberOne{     Emily Longman}
\def \GroupMemberTwo{     Zach Rogers}
\def \GroupMemberThree{     Dominic Giacoppe}
\def \CapstoneProjectName{    Security for Robotics}
\def \CapstoneSponsorCompany{ Oregon State University}
\def \CapstoneSponsorPerson{    Vedanth Narayanan}

% 2. Uncomment the appropriate line below so that the document type works
\def \DocType{    %Problem Statement
        %Requirements Document
        %Technology Review
        %Design Document
        Progress Report
        }
      
\newcommand{\NameSigPair}[1]{\par
\makebox[2.75in][r]{#1} \hfil   \makebox[3.25in]{\makebox[2.25in]{\hrulefill} \hfill    \makebox[.75in]{\hrulefill}}
\par\vspace{-12pt} \textit{\tiny\noindent
\makebox[2.75in]{} \hfil    \makebox[3.25in]{\makebox[2.25in][r]{Signature} \hfill  \makebox[.75in][r]{Date}}}}
% 3. If the document is not to be signed, uncomment the RENEWcommand below
%\renewcommand{\NameSigPair}[1]{#1}

%%%%%%%%%%%%%%%%%%%%%%%%%%%%%%%%%%%%%%%
\begin{document}
\begin{titlepage}
    \pagenumbering{gobble}
    \begin{singlespace}
      \includegraphics[height=4cm]{coe_v_spot1}
        \hfill 
        % 4. If you have a logo, use this includegraphics command to put it on the coversheet.
        %\includegraphics[height=4cm]{CompanyLogo}   
        \par\vspace{.2in}
        \centering
        \scshape{
            \huge CS Capstone \DocType \par
            {\large\today}\par
            \vspace{.5in}
            \textbf{\Huge\CapstoneProjectName}\par
            \vfill
            {\large Prepared for}\par
            \Huge \CapstoneSponsorCompany\par
            \vspace{10pt}
            {\Large\NameSigPair{\CapstoneSponsorPerson}\par}
            {\large Prepared by }\par
            Group\CapstoneTeamNumber\par
            % 5. comment out the line below this one if you do not wish to name your team
            \CapstoneTeamName\par 
            \vspace{10pt}
            {\Large
                \NameSigPair{\GroupMemberOne}\par
                \NameSigPair{\GroupMemberTwo}\par
                \NameSigPair{\GroupMemberThree}\par
            }
            \vspace{20pt}
        }
        \begin{abstract}
          In drones and other networked robotics there is a broad array of security vulnerabilities that can be leveraged in an attack.
          We will evaluate the ROS to find as many of these security holes as we can and document them.
          The different vulnerabilities found will be categorized into malware, sensor hacks, network and control channel attacks, and physical breaches.
          For some of these exploits we may be able to implement solutions, which will also be documented.
          These findings and any solutions will be added to an ongoing academic effort to make robotics more secure.
        \end{abstract}     
    \end{singlespace}
\end{titlepage}
\newpage
\pagenumbering{arabic}
\tableofcontents
% 7. uncomment this (if applicable). Consider adding a page break.
%\listoffigures
%\listoftables
\clearpage

\section{Introduction}
\subsection{Week 0-3}
This is the first blog post. This last week was mostly continuing to hash out details with our client, and figuring out what direction we want to take on the project. Next week should be more or less the same; although we will get to see our probable test drone next week. Only problem this week was figuring out when and where to meet.

Ah yes, the first blog post. We spent time trying to figure out a good time and place to meet with our client, and hasing out details. We got some input from some research professors, which helped us to refine the goals of the project. Next week we hope to get antiquated with the hardware and start planning attack vectors and threat modelling.

The first few weeks of this course were spent analyzing which projects we wanted, who we would be working with, and how we wanted to initially approach our given problem. It was somewhat of an adjustment period to the course and the next nine months. The calm before the storm essentially. These weeks also provided us with an opportunity to prepare for all the technical writing and research we were about to do.

\subsection{Week 4}
This week was mostly about clarification and finding out that we aren't doing our paperwork correctly; also we got to see the drone. The drone runs beaglebone-black stapled to a pixboard, and should run ROS. Also it has 6 propellers, a programmable power supply, and a honking battery. 
As far as goals go, we are more or less set on finding/fixing vulnerabilities in ROS at this point, as it has nice known big ones. It sounds like we might work with the SROS (secure ROS) project, but the SROS project is also in it's infancy and only has some goals on it's page.
Also, we met Jesse, an EX Intel employee with heavy crypto/security experience, and I look forward to learning under him.

We spent time clarifying some logistical things with the class, including getting feedback on our Problem Statement writeup; which we need to redo. We also got to see the drone! There's some work to be done before we can get it working, and we should be able to have it run ROS. Being that we are using ROS, we already have some vulnerabilities that we can look at, as ROS has some out of the box. We may be looking towards SROS, a secure variant of ROS, though it is still a very early project. We also had a chance to meet with Jesse, a crypto researcher who once worked for Intel. It will be awesome having his input on the project as we move further along. Next week we hope to finish up our Problem Statement and get our Requirements doc written.

During this week we continued our work on getting the project better defined and potentially revising our problem statement. We also got to have a great meeting with Rakesh and Jesse, who gave us some really cool professional insight into what we could potentially focus on. We've mostly confirmed that we're using ROS at this point, and we have some ideas about ways to attack it, but nothing has really been solidified yet. We do hope to be able to contribute to the SROS project, but we're not sure it will be usable for us.

\subsection{Week 5}
Most things are temporarily on hold while midterms come. We did finish our draft for our project specification, however. Probably needs revision but we'll wait for the feedback. Still need a team name.

This week we finished up our revised Problem Statement document, and should be squared away on that. We also spent a lot of time trying to work out our requirements document to get our first draft submitted. We were able to meet briefly with our client to check in and make sure everyone was on the same page with regards to the requirements document. After we finish up some formatting and polishing, we should have a good starting point for moving forward.

This week was somewhat of a lull, but we were able to get our problem statement revised, which will help us. We also started work on an initial draft of our requirements document in preparation for the final version due next week. We met as a group to discuss it and have a great plan for moving forward. We also now have seen our drones, but are lacking some components still.

\subsection{Week 6}
Midterms have finally passed us, my grades remain in a solid B range. Life goes on. As far as the project goes, we will be meeting tonight to put the finisher on the project specifications, sign it, and turn it in. We also got access to our drones this week. We will be in charge of fixing one of them (prior student broke something), and adding the parts we will be using (beaglebone black, some legs) before winter.

We had a busy week trying to put together our final version of the Requirements document. We were not able to meet with our client until the end of the week, which caused some concerns regarding our document being too ambiguous. However, being that this is a research project, with requirements that can't fully be defined until we complete our threat model. So we are hoping that we can make changes to this document later on.

Despite our solid first draft we still scrambled to pull together our final version of the requirements document, mainly because it was a little bit unnatural for a research project. There was still so much ambiguity that it seemed like we were pulling at straws and making wild guesses as to what we need. As for the hardware we do now actually have our drones but we need to do some surgery on them and I need to resolder a component on one of the PixHawk boards that Sam broke.

\subsection{Week 7}
Not much this week. Mostly attempting to make progress on the Tech doc, but everyone but me seems to be busy. Other 
classes have more for me to do, so...

This week we made some rough sections for the tech review, but it's been really difficult for us to figure out enough of them that make sense to do a tech review on. The majority of our project is based around the research lifestyle and ROS, but outside of that we don't have a ton to talk about. Hopefully when we have a more specific view of what we're doing we can come back and better define this.

\subsection{Week 8, Fall}
This week was trying to get a tech doc together. Turns out we were all sick and busy except me, which explains why 
nothing got done. This means we are spilling over into design doc time, but that has it's own issues. Still not sure what to write about really. Also we are using a pixhawk, not a beaglebone? 

Things were a bit of a struggle this week and not a lot got done. I've had a cold and Zach has also been sick, so it was mostly just each of us trying to finish our tech doc sections. A lot of other classes have gotten chaotic this week too, so we were definitely not as focused as we needed to be on this.

\subsection{Week 9}
Class is basically over tomorrow so I might as well write now. We met today to discuss how we are going to finish the tech doc, along with further debating some technical aspects of the project, most specifically for me whether or not we will be using SROS or not, and if our board can support the latest version of ROS. We also discussed the preliminary aspects of the design doc, which I will start on about now.

This week was a short one with the holiday, and was spent trying to catch up on the tech doc that both Zach and I failed to complete by the deadline. I've talked with both Kirsten and Kevin about it and they had useful pointers, but it still seems like such a stretch when writing it. We're also starting to think about the design doc coming up next week and the end of term assignments.

\subsection{Week 10}
This was the last week of classes and we focused quite a lot of our effort into making the best design document we could. I'm actually really happy with the outcome, and it was probably the most useful of the documents we've written this term. I do wish that I'd had just a little more time to add a better literature review section per Kirsten's suggestion, but we can still add those sources into our final progress document. That's all we have left to do, other than make the video presentation of it. I can't believe this term is already over.

\section{Retrospective}
Retrospective chart here

\bibliographystyle{IEEEtran}
\bibliography{progress_report.bib}

\end{document}