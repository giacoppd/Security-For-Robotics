\documentclass[IEEEtran,letterpaper,10pt,notitlepage,draftclsnofoot,onecolumn]{article}

\usepackage{nopageno}
\usepackage{alltt}
\usepackage{float}
\usepackage{color}
\usepackage{url}
\usepackage{balance}
\usepackage{enumitem}
\usepackage{pstricks, pst-node}
\usepackage{geometry}
\geometry{textheight=9.5in, textwidth=7in}
\newcommand{\cred}[1]{{\color{red}#1}}
\newcommand{\cblue}[1]{{\color{blue}#1}}
\usepackage{hyperref}
\usepackage{textcomp}
\usepackage{listings}
\usepackage{titling}
\usepackage{graphicx}
\usepackage{url}
\usepackage{setspace}

\definecolor{dkgreen}{rgb}{0,0.6,0}
\definecolor{gray}{rgb}{0.5,0.5,0.5}
\definecolor{mauve}{rgb}{0.58,0,0.82}
\lstset{frame=tb,
  language=c,
  aboveskip=3mm,
  belowskip=3mm,
  showstringspaces=false,
  columns=flexible,
  basicstyle={\small\ttfamily},
  numbers=none,
  numberstyle=\tiny\color{gray},
  keywordstyle=\color{blue},
  commentstyle=\color{dkgreen},
  stringstyle=\color{mauve},
  breaklines=true,
  breakatwhitespace=true,
  tabsize=3
}

% 1. Fill in these details
\def \CapstoneTeamName{   RoboSec}
\def \CapstoneTeamNumber{   50}
\def \GroupMemberOne{     Emily Longman}
\def \GroupMemberTwo{     Zach Rogers}
\def \GroupMemberThree{     Dominic Giacoppe}
\def \CapstoneProjectName{    Security for Robotics}
\def \CapstoneSponsorCompany{ Oregon State University}
\def \CapstoneSponsorPerson{    Vedanth Narayanan}

% 2. Uncomment the appropriate line below so that the document type works
\def \DocType{    %Problem Statement
        %Requirements Document
        %Technology Review
        %Design Document
        Midterm Progress Report
        }

\newcommand{\NameSigPair}[1]{\par
\makebox[2.75in][r]{#1} \hfil   \makebox[3.25in]{\makebox[2.25in]{\hrulefill} \hfill    \makebox[.75in]{\hrulefill}}
\par\vspace{-12pt} \textit{\tiny\noindent
\makebox[2.75in]{} \hfil    \makebox[3.25in]{\makebox[2.25in][r]{Signature} \hfill  \makebox[.75in][r]{Date}}}}
% 3. If the document is not to be signed, uncomment the RENEWcommand below
\renewcommand{\NameSigPair}[1]{#1}

%%%%%%%%%%%%%%%%%%%%%%%%%%%%%%%%%%%%%%%
\begin{document}
\begin{titlepage}
    \pagenumbering{gobble}
    \begin{singlespace}
      \includegraphics[height=4cm]{coe_v_spot1}
        \hfill
        % 4. If you have a logo, use this includegraphics command to put it on the coversheet.
        %\includegraphics[height=4cm]{CompanyLogo}
        \par\vspace{.2in}
        \centering
        \scshape{
            \huge CS Capstone \DocType \par
            {\large\today}\par
            \vspace{.5in}
            \textbf{\Huge\CapstoneProjectName}\par
            \vfill
            {\large Prepared for}\par
            \Huge \CapstoneSponsorCompany\par
            \vspace{10pt}
            {\Large\NameSigPair{\CapstoneSponsorPerson}\par}
            {\large Prepared by }\par
            Group\CapstoneTeamNumber\par
            % 5. comment out the line below this one if you do not wish to name your team
            \CapstoneTeamName\par
            \vspace{10pt}
            {\Large
                \NameSigPair{\GroupMemberOne}\par
                \NameSigPair{\GroupMemberTwo}\par
                \NameSigPair{\GroupMemberThree}\par
            }
            \vspace{20pt}
        }
        \begin{abstract}
          In drones and other networked robotics there is a broad array of security vulnerabilities that can be leveraged in an attack.
          We will evaluate the ROS to find as many of these security holes as we can and document them.
          The different vulnerabilities found will be categorized into malware, sensor hacks, network and control channel attacks, and physical breaches.
          For some of these exploits we may be able to implement solutions, which will also be documented.
          These findings and any solutions will be added to an ongoing academic effort to make robotics more secure.
        \end{abstract}
    \end{singlespace}
\end{titlepage}
\newpage
\pagenumbering{arabic}
\tableofcontents
% 7. uncomment this (if applicable). Consider adding a page break.
%\listoffigures
%\listoftables
\clearpage

\section{Overview}
It has finally come time to harvest the fruits of this project and turn them into usable research data.
After a large amount of addition to the code base of this project in the past six weeks, the available data has been broadened with a wider variety of attack types.
When the packages cover more ground, better and more generalizable conclusions can be drawn, which is the overarching purpose of this project.

This purpose was to find vulnerabilities in ROS, the Robot Operating System, and then prove their existence by writing code to exploit them. 
It was quickly discovered that ROS is simply is not designed to be secure by most standards and as such that finding vulnerabilities is akin to shooting fish in a barrel. 
This also made it difficult to follow through on one of our original stretch goals of patching any found vulnerabilities, as they tend to stem from a design choice than from any sort of coding error. 
Overall the project went from expecting to find one or two vulnerabilities to seeing how many interesting and unique ways we could break ROS, but the end goal has remained the same: create ROS packages to show the existence of vulnerabilities in ROS. 

%describle more of the big picture of the project

\section{Term Progress}
The largest portion of progress this term has been made within the codebase.
New exploits in the realms of authentication, network protocols, and the operating system have been made, and documentation for all packages has been updated.
After changes in the plan for this term were made in the beginning, the work on the drone was scrapped and production of exploitation code was increased.
Some initial dissection of data was also done for the expo poster, which provided a start for the metrics on the final paper.
This mainly involved using the FMEA variables to rank the importance of each exploit and compare their failure modes.

\section{Dominic's Progress}
Dominic developed multiple packages that exploited the lack of process pre-checking
in ROS packages. ROS does not check to ensure that packages operate in a safe manner before running
them by design, and as such any package run by ROS can quite easily do unsafe operations. The creators of 
ROS simply assumed that all packages run in ROS would all be run intentionally and benevolent in nature,
but the fact remains that there is no check on packages run by ROS to ensure good behavior. In particular,
ROS packages can make system calls to the underlying operating system's shell, which in turn gives them full 
run of the robot with the right commands. This discovery was made about 2 weeks into winter term, as it is
technically intentional functionality and documented; but it is very unsafe. Dominic was able to use packages
to dump the entire root directory of the robot to another system, turn off the robot's wireless networking 
capabilites, kill all ROS processes, and other problematic actions. Dominic also attempted to create a TCP node
that would trick a ROS publisher node into publishing it's data to the TCP node, but was unable to due to 
technical limitations.
%discuss individual progress

\section{Research Documentation}
%Whitepaper draft goes here!

\section{Conclusion}
Things really started to come together and form a thorough research base.
As described above the final exploits cover both breadth and depth of the system and can be analyzed in a variety of ways.

%add recaps from whitepaper

\bibliographystyle{IEEEtran}
\bibliography{midterm_progress.bib}

\end{document}
