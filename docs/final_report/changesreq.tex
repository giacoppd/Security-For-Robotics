\documentclass[IEEEtran,letterpaper,10pt,notitlepage,draftclsnofoot,onecolumn]{article}

\usepackage{nopageno}
\usepackage{alltt}
\usepackage{float}
\usepackage{color}
\usepackage{balance}
\usepackage{enumitem}
\usepackage{pstricks, pst-node}
\usepackage{geometry}
\geometry{textheight=9.5in, textwidth=7in}
\newcommand{\cred}[1]{{\color{red}#1}}
\newcommand{\cblue}[1]{{\color{blue}#1}}
\usepackage{hyperref}
\usepackage{textcomp}
\usepackage{listings}
\usepackage{titling}
\usepackage{graphicx}
\usepackage{url}
\usepackage{setspace}

\definecolor{dkgreen}{rgb}{0,0.6,0}
\definecolor{gray}{rgb}{0.5,0.5,0.5}
\definecolor{mauve}{rgb}{0.58,0,0.82}
\lstset{frame=tb,
  language=python,
  aboveskip=3mm,
  belowskip=3mm,
  showstringspaces=false,
  columns=flexible,
  basicstyle={\small\ttfamily},
  numbers=none,
  numberstyle=\tiny\color{gray},
  keywordstyle=\color{blue},
  commentstyle=\color{dkgreen},
  stringstyle=\color{mauve},
  breaklines=true,
  breakatwhitespace=true,
  tabsize=3
}
\begin{document}

\section{Changes to the requirements}
The only explicit change we had to our requirements was that our team used 
ROS Indigo Igloo over Kinetic Kame, as a large part of the existing ROS codebase
was created during Indigo and Kame had some compatibility issues with some 
Indigo-based code. There was also the understanding that we would not be 
producing patches to patch any vulnerabilities we found, as all of them
stem from underlying design choices in ROS itself rather than programming
errors. This makes everything more or less unpatchable, as it would require
design changes to ROS itself that are simply not in consideration by the 
ROS development team. In turn, since our team produced no patches, we did 
not send them to the SROS project for consideration either.
\end{document}
