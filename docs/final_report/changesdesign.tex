\documentclass[IEEEtran,letterpaper,10pt,notitlepage,draftclsnofoot,onecolumn]{article}

\usepackage{nopageno}
\usepackage{alltt}
\usepackage{float}
\usepackage{color}
\usepackage{balance}
\usepackage{enumitem}
\usepackage{pstricks, pst-node}
\usepackage{geometry}
\geometry{textheight=9.5in, textwidth=7in}
\newcommand{\cred}[1]{{\color{red}#1}}
\newcommand{\cblue}[1]{{\color{blue}#1}}
\usepackage{hyperref}
\usepackage{textcomp}
\usepackage{listings}
\usepackage{titling}
\usepackage{graphicx}
\usepackage{url}
\usepackage{setspace}
\usepackage{fancyhdr}
\pagestyle{fancy}
\fancyfoot[C]{DESIGNDELTA--\thepage}
\begin{document}
\LARGE\textbf{Changes to the project design\\ \\}
\normalsize
We had 2 major changes from our original design: we did not end up using 
the drone at all, and we determined that SROS was not viable for a few 
reasons. The drone simply took too much of our time to get it working to be 
usable for the project. The team initially thought making the drone usable 
for our purposes would take about a week, by the time we decided to drop it 
from the project it was almost the end of winter term. (ZACH, fill in your 
drone struggles here, only like 2 sentences though save the big bash for 
the failures bit). SROS was dropped much earlier on in the project when 
the team discovered that the prefered installation method for it was 
containers, which the team had no experience with and might not be feasible 
on the drone, and that it was so poorly documented that the team did not feel 
that it was robust enough to be a working alternative to ROS. Overall, SROS 
was not determined to be worth the amount of extra overhead to get it 
operational for our purposes, and was subsequently cut from the project.
\end{document}
