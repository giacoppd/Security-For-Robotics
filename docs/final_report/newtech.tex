\documentclass[IEEEtran,letterpaper,10pt,notitlepage,draftclsnofoot,onecolumn]{article}

\usepackage{nopageno}
\usepackage{alltt}
\usepackage{float}
\usepackage{color}
\usepackage{balance}
\usepackage{enumitem}
\usepackage{pstricks, pst-node}
\usepackage{geometry}
\geometry{textheight=9.5in, textwidth=7in}
\newcommand{\cred}[1]{{\color{red}#1}}
\newcommand{\cblue}[1]{{\color{blue}#1}}
\usepackage{hyperref}
\usepackage{textcomp}
\usepackage{listings}
\usepackage{titling}
\usepackage{graphicx}
\usepackage{url}
\usepackage{setspace}
\usepackage{fancyhdr}
\pagestyle{fancy}
\fancyfoot[C]{LEARN--\thepage}
\begin{document}
\LARGE\textbf{Learning Resources\\ \\}\normalsize
The defacto outside resource for this project was the ROS wiki\cite{ROSWIKI}(wiki.ros.org), as it
contains all the developer written documentation for ROS itself. It also
contains some tutorials and descriptions of common tasks in ROS that the
team refered to while developing packages.\\
For some Linux and operating system related questions, Stack Overflow\cite{SO}(stackoverflow.com)
continues to be a semi-reliable source for getting things working in short order, and
for some of the operating system packages it was refered to in order to answer 
various questions.\\
--RESEARCH GOES HERE--
--FOR THE LOVE OF GOD GUYS FILL THIS IN--




HEREEEEEEEEEEEEEEEEEEEEEEE




--i hope there is a lot--
--Emily: Put that threat model book here?--\\
Lastly, our team had an abundance of on campus resources. Kevin McGrath was our go-to
concerning attempting to get the drone operational, at least until the team gave up on
that. Kirsten Winters was crucial in our document and research design, along with Jonathan 
Dodge for some general steering and TeX related issues. And of course, our client Vedanth 
Narayanan was a great resource, along with generally steering the project to his liking.
\end{document}
