\documentclass[IEEEtran,letterpaper,10pt,notitlepage,draftclsnofoot,onecolumn]{article}

\usepackage{nopageno}
\usepackage{alltt}
\usepackage{float}
\usepackage{color}
\usepackage{balance}
\usepackage{enumitem}
\usepackage{pstricks, pst-node}
\usepackage{geometry}
\geometry{textheight=9.5in, textwidth=7in}
\newcommand{\cred}[1]{{\color{red}#1}}
\newcommand{\cblue}[1]{{\color{blue}#1}}
\usepackage{hyperref}
\usepackage{textcomp}
\usepackage{listings}
\usepackage{titling}
\usepackage{graphicx}
\usepackage{url}
\usepackage{setspace}
\usepackage{fancyhdr}
\pagestyle{fancy}
\fancyfoot[C]{LEARN--\thepage}
\begin{document}
\LARGE\textbf{Learning Resources\\ \\}\normalsize
The defacto outside resource for this project was the ROS wiki\cite{ROSWIKI}(wiki.ros.org), as it
contains all the developer written documentation for ROS itself. It also
contains some tutorials and descriptions of common tasks in ROS that the
team referred to while developing packages.\\
For some Linux and operating system related questions, Stack Overflow\cite{SO}(stackoverflow.com)
continues to be a reliable source for getting things working in short order, and
for some of the operating system packages it was referred to in order to answer
various questions.\\
There were also a lot of pre-existing research that helped to guide our project, and pointed us in the right
direction regarding how to tackle writing exploit packages for ROS. Most of these can be found in our Github
repo \cite{GitRepoSupporting}(https://github.com/ZachR0/Security-For-Robotics/tree/master/docs/supporting\_research) or discussed in the research paper.

The ArduPilot website and documentation was also helpful with regards to getting the Beagle Bone Black and
Pixhawk v1.6 Cape flight controller interfaced and talking, though also showed that we really should have been
using the newer flight controllers and not the outdated Pixhawk v1.6 \cite{ArduProjectDevDocs}(http://ardupilot.org/dev/docs/building-for-beaglebone-black-on-linux.html). This
documentation also helped us figure out how to wire all the components on our drone and get the Mission Planner software configured \cite{ArduProjectDocs}(http://ardupilot.org/copter/index.html).

When it comes to books, we only really used one hard copy book, which was for the development of our Threat Models.
This book was called textit{Threat Modeling, Designing for Security} by Adam Shostack. \cite{TMDS}
All other reference sources used were journals or websites, and these are discussed more extensively in the research paper.

Lastly, our team had an abundance of on campus resources. Kevin McGrath was our go-to
concerning attempting to get the drone operational, at least until the team gave up on
that. Dave Nevin was also a great resource helping to point us in the right direction while making sure
we operated within OSU's Security Policy.
Kirsten Winters was crucial in our document and research design, along with Jonathan
Dodge for some general steering and TeX related issues. And of course, our client Vedanth
Narayanan was a great resource, along with generally steering the project to his liking.

\bibliographystyle{IEEEtran}
\bibliography{newtech}
\end{document}
