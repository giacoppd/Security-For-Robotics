\documentclass[IEEEtran,letterpaper,10pt,notitlepage,draftclsnofoot,onecolumn]{article}

\usepackage{nopageno}
\usepackage{alltt}
\usepackage{float}
\usepackage{color}
\usepackage{balance}
\usepackage{enumitem}
\usepackage{pstricks, pst-node}
\usepackage{geometry}
\geometry{textheight=9.5in, textwidth=7in}
\usepackage{hyperref}
\usepackage{textcomp}
\usepackage{listings}
\usepackage{titling}
\usepackage{graphicx}
\usepackage{url}
\usepackage{setspace}
\usepackage{fancyhdr}
\pagestyle{fancy}
\fancyfoot[C]{LEARNED--\thepage}
\begin{document}

\section{What Did We Learn?}
\subsection{Dominic}
As far as technical information is concerned, I (re)learned many
things about ROS and Linux, in particular creating and building 
ROS packages along with ROS process control and the publisher/subscriber 
system. However, I feel that the real learning for me was in 
non-technical areas, particularly communication and scheduling. First 
and foremost this project reinforced the idea that if I’m producing 
work for someone, they should see it before the deadline, so they 
can tell me what I missed ahead of time instead of after. This was less 
of an issue for me personally then for the team as a whole, especially 
with our documentation, but the point stands. Our whitepaper was not 
up to snuff, nor our Spring midterm documentation, nor our design document 
and tech review, not even the blog posts apparently, and all of that could 
have been avoided if someone had looked it over before submission and told 
us we were completely off base. However, none of this happened because we 
tended to cut real close to the deadline on all of these documents, hence 
scheduling: starting in earnest ASAP instead of half-heartedly starting it 
and only really getting to it with 2 days to spare would have made it much 
easier to present a full draft to those in charge before the deadline instead 
of just getting it submitted with hours to spare. Much of the delay for those 
papers had to do with factors outside of my control, which was unfortunate, but 
the point remains that better planning could have helped with this. The biggest 
thing I learned about project management is that if you aren’t managing the 
project, it’s best to assume nobody is until evidence is provided otherwise. 
Most of spring term felt like I was attempting to get everyone involved on the 
same page trying insure the team knew what was expected of us and occasionally 
moonlighting as a ghostwriter for Emily. On that note, I learned a lot about 
project management, or at least the interpersonal communication part, not so 
much the planning part as by spring term it was too late to plan. Ironically, 
this played out like it normally does for group work for me: I’m in charge of 
confirming requirements and relaying information between my group and those on 
high, and as long as I get my share of the work done things normally play out 
alright from there. If I got a redo, I’d first and foremost redirect this
project away from research, as I still can’t justify this as a research paper.
I would probably just make sure Vee nipped the threat model thing in the bud 
before it became the purpose of the project, as neither of us expected it to 
work out this way. Next, I would have never bothered with the drone, as I doubt 
it would have been useful even if we had gotten it to work and it took way too 
much of Zach’s time. Last, I would have tried to look for a heartbleed-esqe bug 
in the ROS subscriber-publisher system for arbitrary code execution, because I 
think that’s the real holy grail as far as OS based attacks. Everything I did 
relied on already having the ability to execute arbitrary code on the ROS 
system, which is a rather large assumption to make.

\subsection{Emily}
This project taught me more than I ever thought it would in just about every way.
On the technical side I learned first and foremost about the core of our project, ROS.
I had absolutely no robotics experience going into this so it was a fantastic opportunity to learn about a whole new field and study it in depth.
I also hadn't had much experience with drones before this, so it was really cool to be able to play with some high quality hardware and learn how it worked.
I did have security knowledge coming in, and applied it vastly in the creation of our exploits, but in writing each one of them I gained quite a bit of new knowledge about the portion of the system with which I was working.
Being able to gain all of this technical knowledge was amazing, but I think the most important learning was in the more non-technical realm.

I've always loved research and wanted to do it on a large scale, but I truly underestimated how difficult it is to do on a project longer than a term.
I really had to learn (by failing time and time again) how to best manage our time and expectations.
Naively I thought we would be able to accomplish a lot more, and do it much more smoothly and quickly.
One of the most difficult lessons of this whole project for me was learning how to ensure that I was serving the expectations of everyone and their requirements equally, and not just focusing on what I considered to be the most important.
It's difficult for most people to get serious work done on things that they don't think matter as much as something else, and I learned a great deal about trying to balance that.

On smaller scales I've often ended up managing group projects so I somewhat fell into that role in this, but it was a lot harder to juggle with this one.
Time management was challenging, and keeping track of everything at once made this even more difficult, but after each term I felt I had gotten better at it.
It was also clear that knowing communication is crucial and actually communicating effectively are two different things, with it being much easier said than done.
If I had to say I learned the most about anything from this project, it was about communication in project management and how difficult it can be when not done perfectly.

If I had to do it all again, I would have everyone meeting much more regularly, and working together as a group in the same room.
Of course having varied and busy schedules made this tough, but we could have done better.
I would also have started just about everything sooner. 
Humans are notoriously bad at time estimation, and this project truly proved that to me.
I'm still happy with the experience I had though, because it taught me so much, from technical details to collaboration.
While I may not have loved it at many points in time, I greatly appreciate my capstone experience in retrospect because I needed the challenges it provided.

\subsection{Zach}

\end{document}
