\documentclass[IEEEtran,letterpaper,10pt,notitlepage,draftclsnofoot,onecolumn]{article}

\usepackage{nopageno}
\usepackage{alltt}
\usepackage{float}
\usepackage{color}
\usepackage{balance}
\usepackage{enumitem}
\usepackage{pstricks, pst-node}
\usepackage{geometry}
\geometry{textheight=9.5in, textwidth=7in}
\newcommand{\cred}[1]{{\color{red}#1}}
\newcommand{\cblue}[1]{{\color{blue}#1}}
\usepackage{hyperref}
\usepackage{textcomp}
\usepackage{listings}
\usepackage{titling}
\usepackage{graphicx}
\usepackage{url}
\usepackage{setspace}

\definecolor{dkgreen}{rgb}{0,0.6,0}
\definecolor{gray}{rgb}{0.5,0.5,0.5}
\definecolor{mauve}{rgb}{0.58,0,0.82}
\lstset{frame=tb,
  language=python,
  aboveskip=3mm,
  belowskip=3mm,
  showstringspaces=false,
  columns=flexible,
  basicstyle={\small\ttfamily},
  numbers=none,
  numberstyle=\tiny\color{gray},
  keywordstyle=\color{blue},
  commentstyle=\color{dkgreen},
  stringstyle=\color{mauve},
  breaklines=true,
  breakatwhitespace=true,
  tabsize=3
}
\begin{document}
Dominic:
As far as technical information is concerned, I (re)learned many
things about ROS and Linux, in particular creating and building 
ROS packages along with ROS process control and the publisher|subscriber 
system. However, I feel that the real learning for me was in 
non-technical areas, particularly communication and scheduling. First 
and foremost this project reinforced the idea that if I’m producing 
work for someone, they should see it before the deadline, so they 
can tell me what I missed ahead of time instead of after. This was less 
of an issue for me personally then for the team as a whole, especially 
with our documentation, but the point stands. Our whitepaper was not 
up to snuff, nor our Spring midterm documentation, nor our design document 
and tech review, not even the blog posts apparently, and all of that could 
have been avoided if someone had looked it over before submission and told 
us we were completely off base. However, none of this happened because we 
tended to cut real close to the deadline on all of these documents, hence 
scheduling: starting in earnest ASAP instead of half-heartedly starting it 
and only really getting to it with 2 days to spare would have made it much 
easier to present a full draft to those in charge before the deadline instead 
of just getting it submitted with hours to spare. Much of the delay for those 
papers had to do with factors outside of my control, which was unfortunate, but 
the point remains that better planning could have helped with this. The biggest 
thing I learned about project management is that if you aren’t managing the 
project, it’s best to assume nobody is until evidence is provided otherwise. 
Most of spring term felt like I was attempting to get everyone involved on the 
same page trying insure the team knew what was expected of us and occasionally 
moonlighting as a ghostwriter for Emily. On that note, I learned a lot about 
project management, or at least the interpersonal communication part, not so 
much the planning part as by spring term it was too late to plan. Ironically, 
this played out like it normally does for group work for me: I’m in charge of 
confirming requirements and relaying information between my group and those on 
high, and as long as I get my share of the work done things normally play out 
alright from there. If I got a redo, I’d first and foremost redirect this
project away from research, as I still can’t justify this as a research paper.
I would probably just make sure Vee nipped the threat model thing in the bud 
before it became the purpose of the project, as neither of us expected it to 
work out this way. Next, I would have never bothered with the drone, as I doubt 
it would have been useful even if we had gotten it to work and it took way too 
much of Zach’s time. Last, I would have tried to look for a heartbleed-esk bug 
in the ROS subscriber-publisher system for arbitrary code execution, because I 
think that’s the real holy grail as far as OS based attacks. Everything I did 
relied on already having the ability to execute arbitrary code on the ROS 
system, which is a rather large assumption to make.
